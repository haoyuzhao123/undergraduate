%Homework template created by Haoyu Zhao, Oct 23,2017
\documentclass{article}

    \usepackage{ctex}
    \usepackage{fontspec}
    
    %the math packages
    \usepackage{amsmath,amsfonts,amssymb,amsthm}
    \theoremstyle{plain}
    \newtheorem{thm}{Theorem}[section]
    \newtheorem{lem}[thm]{Lemma}
    \newtheorem{prop}[thm]{Proposition}
    \newtheorem*{cor}{Corollary}
    
    \theoremstyle{definition}
    \newtheorem{defn}{Definition}[section]
    \newtheorem{conj}{Conjecture}[section]
    \newtheorem{exmp}{Example}[section]
    
    \theoremstyle{remark}
    \newtheorem*{rem}{Remark}
    \newtheorem*{note}{Note}
    
    \usepackage{mathtools}
    \usepackage{optidef}
    %add new math packages here
    
    %--------------------------
    
    
    %the algorithm packages
    \usepackage{algorithm}
    \usepackage{algorithmicx}
    \usepackage{algpseudocode}
    \algrenewcommand\algorithmicrequire{\textbf{Input:}}
    \algrenewcommand\algorithmicensure{\textbf{Output:}}
    %add new algorithm packages here
    
    %-------------------------------
    
    \usepackage{graphicx}
    \usepackage{tikz}
    \usepackage{subfigure}
    
    \usepackage{listings}
    
    % In case you need to adjust margins:
    \topmargin=-0.45in      %
    \evensidemargin=0in     %
    \oddsidemargin=0in      %
    \textwidth=6.5in        %
    \textheight=9.0in       %
    \headsep=0.25in         %
    
    %newcommand for the format of the homework
    \newcommand{\Answer}{\ \\\textbf{Answer:} }
    \newcommand{\Acknowledgement}[1]{\ \\{\bf Acknowledgement:} #1}
    
    \newcommand\numberthis{\addtocounter{equation}{1}\tag{\theequation}}
    %end the newcommand for this part
    
    %new command for the partial derivatives
    \newcommand{\pd}[2]{\frac{\partial #1}{\partial #2}}
    \newcommand{\spd}[3]{\frac{\partial^2 #1}{\partial #2 \partial #3}}
    \newcommand{\grad}[1]{\nabla #1}
    \newcommand{\curl}[1]{\nabla \times #1}
    \newcommand{\dive}[1]{\nabla \cdot #1}
    %end the new command for this part
    
    \title{\bf\huge malloc lab实验报告}
    \author{赵浩宇,2016012390,计科60}
    
    \begin{document}
    
    \maketitle

    \section{实验要求}
    本次实验中,要求实现一个动态存储分配的一个C程序,实现malloc, free和realloc三个部分。

    \section{算法概述}
    首先最简单的是根据课本上的代码,进行复现。之后再将课本上的程序进行修改,修改成为显式空闲链表的实现。之后再对realloc进行优化,即如果当先块后面是一个空闲块,而且当前块与后面空闲块的总大小可以进行realloc,则不进行新的alloc,直接返回原来的指针。最后根据测试样例的形式,根据数据进行相应的优化。

    \section{详细实现描述}
    \subsection{对课本代码进行复现}
    这一个部分比较简单,就是复用书本上的代码,最后实现一点点的辅助函数即可。但是可以发现,书本上面宏的定义让后面针对`地址'的操作变得清晰了许多。对书本上的方法进行了实现之后,用测评程序可以得到大概50分。

    \subsection{加入显式空闲链表}
    我们使用双向链表作为显示空闲链表,算法就是,讲所有空闲块连成链表的数据结构,每次新加入一个空闲块的时候,都将其放在最链表的最前面,当有一个空闲块被使用后,就将这个节点在空闲链表中删除。如果一个空闲块的空间减小,那么我的实现方法是首先将原来的空闲块从链表中删除,之后在链表头部加入一个空间比原来小的空闲链表。

    至于对显式空闲链表的实现,我新建了一个全局变量(unsigned int)作为显式空闲链表头部的地址,之后封装了插入节点和删除节点的辅助函数。其他的地方就是调用辅助函数即可,但是需要注意提前设置`指针'。实现完成后测试分数大概为83分。

    \subsection{针对realloc进行优化}
    原本realloc的实现方法是先进行一次malloc,再将原来的指针free掉,这种实现方式看起来就没有进行优化,所以我针对一种特定的情况进行优化,即如果需要realloc的块后面的跟着一个空闲块,并且两块的总大小比需要真正重新分配的空间要大,那么就不再重新malloc了,直接将原来的非空闲块即可。

    实现上与malloc基本类似,也是根据剩余的大小进行分类,如果剩余空间比较小,那么就不再将空闲空间分开,如果还有较大的剩余空间,那么将空闲的区域分开。加入realloc优化后,测试分数从83分变为85分。

    \subsection{针对数据进行优化}
    经过跟同学的讨论,发现如果对数据进行优化可以有小幅度的提高,即如果malloc的大小在64-128之间,那么将分配的大小变为128,这样可以对测试分数有小幅度的提高,最终测试分数为87分。

    \section{思考与讨论}
    通过malloc lab的实验,我更加深入的了解了动态内存分配的内容,并且知道了使用一些宏定义可以让程序变得简洁清晰,也改善了我的代码习惯。

\end{document}