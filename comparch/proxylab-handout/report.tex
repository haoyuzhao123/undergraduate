%Homework template created by Haoyu Zhao, Oct 23,2017
\documentclass{article}

    \usepackage{ctex}
    \usepackage{fontspec}
    
    %the math packages
    \usepackage{amsmath,amsfonts,amssymb,amsthm}
    \theoremstyle{plain}
    \newtheorem{thm}{Theorem}[section]
    \newtheorem{lem}[thm]{Lemma}
    \newtheorem{prop}[thm]{Proposition}
    \newtheorem*{cor}{Corollary}
    
    \theoremstyle{definition}
    \newtheorem{defn}{Definition}[section]
    \newtheorem{conj}{Conjecture}[section]
    \newtheorem{exmp}{Example}[section]
    
    \theoremstyle{remark}
    \newtheorem*{rem}{Remark}
    \newtheorem*{note}{Note}
    
    \usepackage{mathtools}
    \usepackage{optidef}
    %add new math packages here
    
    %--------------------------
    
    
    %the algorithm packages
    \usepackage{algorithm}
    \usepackage{algorithmicx}
    \usepackage{algpseudocode}
    \algrenewcommand\algorithmicrequire{\textbf{Input:}}
    \algrenewcommand\algorithmicensure{\textbf{Output:}}
    %add new algorithm packages here
    
    %-------------------------------
    
    \usepackage{graphicx}
    \usepackage{tikz}
    \usepackage{subfigure}
    
    \usepackage{listings}
    
    % In case you need to adjust margins:
    \topmargin=-0.45in      %
    \evensidemargin=0in     %
    \oddsidemargin=0in      %
    \textwidth=6.5in        %
    \textheight=9.0in       %
    \headsep=0.25in         %
    
    %newcommand for the format of the homework
    \newcommand{\Answer}{\ \\\textbf{Answer:} }
    \newcommand{\Acknowledgement}[1]{\ \\{\bf Acknowledgement:} #1}
    
    \newcommand\numberthis{\addtocounter{equation}{1}\tag{\theequation}}
    %end the newcommand for this part
    
    %new command for the partial derivatives
    \newcommand{\pd}[2]{\frac{\partial #1}{\partial #2}}
    \newcommand{\spd}[3]{\frac{\partial^2 #1}{\partial #2 \partial #3}}
    \newcommand{\grad}[1]{\nabla #1}
    \newcommand{\curl}[1]{\nabla \times #1}
    \newcommand{\dive}[1]{\nabla \cdot #1}
    %end the new command for this part
    
    \title{\bf\huge proxy lab实验报告}
    \author{赵浩宇,2016012390,计科60}
    
    \begin{document}
    
    \maketitle

    \section{实验要求}
    本次实验中,要求编写一个缓存Web对象的简单Http代理。

    \section{算法概述}
    主要分为三部分,第一个部分实现一个顺序执行的Http代理,第二个部分对第一部分进行拓展,实现一个多线程的Http代理,第三个部分实现本地缓存。

    \section{详细实现描述}
    \subsection{顺序执行Http代理}
    这个部分不是非常的难,但是很。首先要实现对请求的字符串进行分析,需要实现一个parse函数,还需要构建header。在doit函数中,感觉就是按部就班的去把每一个需要实现的功能实现出来,但是错误处理是比较麻烦的。

    \subsection{多线程Http代理}
    这个部分是最简单的,只需要加入thread函数,在函数内分离当前线程,使得执行完毕之后被自动回收即可。

    \subsection{本地缓存Http代理}
    这一部分感觉上是比较麻烦的,但是可能由于测例的原因,要想通过测试,并不用写得非常的完美。我采用的方法是首先将每一块cache block抽象成一个结构体,之后再将cache block的数组抽象成cache结构体。cache的替换策略就采用LRU。但是一个比较重要的问题是,当多线程的时候,如果不加sem,那么读和写都是非常不安全的,这样可能会两个页面都写到一个cache block中去,也有可能有线程在同时读写,这非常不安全。我的实现方法是,对每一个cache block加一个锁,不管是读还是写都只能有一个线程进行操作。同时对于全局计数器counter,也用一个信号量实现互斥。这样安全性可以得到保证,但是性能上肯定是有些慢的。
    
    \section{思考与讨论}
    通过这个实验,我更加深入地理解了IO,网络,以及并发编程的相关内容,我的cache的实现感觉依然可以改的更好,比如说读和写是互斥的,但是可以允许多个进程来读。
    

\end{document}