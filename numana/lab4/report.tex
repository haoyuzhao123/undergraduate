%Homework template created by Haoyu Zhao, Oct 23,2017
\documentclass{article}

    \usepackage{ctex}
    \usepackage{fontspec}
    
    %the math packages
    \usepackage{amsmath,amsfonts,amssymb,amsthm}
    \theoremstyle{plain}
    \newtheorem{thm}{Theorem}[section]
    \newtheorem{lem}[thm]{Lemma}
    \newtheorem{prop}[thm]{Proposition}
    \newtheorem*{cor}{Corollary}
    
    \theoremstyle{definition}
    \newtheorem{defn}{Definition}[section]
    \newtheorem{conj}{Conjecture}[section]
    \newtheorem{exmp}{Example}[section]
    
    \theoremstyle{remark}
    \newtheorem*{rem}{Remark}
    \newtheorem*{note}{Note}
    
    \usepackage{mathtools}
    \usepackage{optidef}
    %add new math packages here
    
    %--------------------------
    
    
    %the algorithm packages
    \usepackage{algorithm}
    \usepackage{algorithmicx}
    \usepackage{algpseudocode}
    \algrenewcommand\algorithmicrequire{\textbf{Input:}}
    \algrenewcommand\algorithmicensure{\textbf{Output:}}
    %add new algorithm packages here
    
    %-------------------------------
    
    \usepackage{graphicx}
    \usepackage{tikz}
    \usepackage{subfigure}
    
    \usepackage{listings}
    
    % In case you need to adjust margins:
    \topmargin=-0.45in      %
    \evensidemargin=0in     %
    \oddsidemargin=0in      %
    \textwidth=6.5in        %
    \textheight=9.0in       %
    \headsep=0.25in         %
    
    %newcommand for the format of the homework
    \newcommand{\Answer}{\ \\\textbf{Answer:} }
    \newcommand{\Acknowledgement}[1]{\ \\{\bf Acknowledgement:} #1}
    
    \newcommand\numberthis{\addtocounter{equation}{1}\tag{\theequation}}
    %end the newcommand for this part
    
    %new command for the partial derivatives
    \newcommand{\pd}[2]{\frac{\partial #1}{\partial #2}}
    \newcommand{\spd}[3]{\frac{\partial^2 #1}{\partial #2 \partial #3}}
    \newcommand{\grad}[1]{\nabla #1}
    \newcommand{\curl}[1]{\nabla \times #1}
    \newcommand{\dive}[1]{\nabla \cdot #1}
    %end the new command for this part
    
    \title{\bf\huge 数值分析实验四}
    \author{赵浩宇,2016012390,计科60}
    
    \begin{document}
    
    \maketitle
    
    \section{实验要求}
    \textbf{1.} 用不同数值方法计算积分$I = \int_0^1 e^xdx$。首先使用复合梯形公式与复合辛普森公式,通过预想估计步长大小,然后根据估计的步长进行程序实验。之后用龙贝格积分计算,如果达到相同精度,许多将区间分为多少等份。

    \textbf{2.} 用复合高斯公式对$4\int_0^1 \frac{1}{1+x^2}$ 做近似积分。是对步长$h$作先验估计,然后根据高斯复合公式进行近似积分,最后将理论值$\pi$与实验结果进行比较。
    
    \section{算法描述}
    \textbf{1.} 对于复合梯形求积分方法,在对步长进行先验估计之后,通过给定的公式
    \[I \approx \frac{h}{2}\sum_{i=0}^{n-1}\left(f(x_i) + f(x_{i+1})\right),\]
    进行计算即可。

    \textbf{2.} 对于复合辛普森求积分方法,在对步长进行先验估计之后,通过给定的公式
    \[I \approx \frac{h}{6}\sum_{i=0}^{n-1}\left(f(x_i) + f(x_{i+1}) + 4f\left(\frac{x_i+x_{i+1}}{2}\right)\right),\]
    进行计算即可。

    \textbf{3.} 对于龙贝格求积分方法,首先依据公式计算$T_0^{(k)}$, 其中
    \[T_0^{(k)} = \frac{1}{2}T_0^{(k-1)} + \frac{h}{2}\sum_{i=0}^{2^{k-1}-1}f(x_{i+\frac{1}{2}}).\]
    之后再根据外推法的公式,计算$T_k^{(0)}$,其中计算公式为
    \[T_m^{(k)} = \frac{4^m}{4^m-1}T_{m-1}^{(k+1)} - \frac{1}{4^m-1}T_{m-1}^{(k)}.\]
    如果$|T_k^{0} - T_{k-1}^0|<\epsilon$,则终止计算,所得结果$I\approx T_k^0$。

    \textbf{4.} 对于高斯求积公式,在对步长进行先验估计之后,根据给定的积分公式,
    \[I \approx \frac{h}{2}\sum_{i=0}^{n-1}\left(f(x_i) + f(x_{i+1}\right),\]
    进行计算即可。
    
    
    \section{程序清单}
    详细程序清单见表\ref{list}。
    \begin{table}[!htbp]
        \centering
        \begin{tabular}{c|c}
            \hline
            \hline
            文件名称 & 作用 \\
            \hline
            code.cc & 实验代码\\
            integral.out & 实验结果输出文件\\
            report.tex & 实验报告源码\\
            report.pdf & 实验报告\\
            \hline
            \hline
        \end{tabular}
        \caption{程序清单}
        \label{list}
    \end{table}

    \section{运行结果}
    首先根据复合梯形公式的余项,
    \[R_n(f) = -\frac{b-a}{12}h^2f''(\eta),\]
    复合辛普森公式的余项,
    \[R_n(f) = -\frac{b-a}{180}\left(\frac{h}{2}\right)^4f^{(4)}(\eta),\]
    以及高斯求积分公式的余项
    \[R_n(f) = \frac{(b-a)h^4}{4320}f^{(4)}(\eta),\]
    经过计算,得到对于复合梯形公式需要$n=476$,对于复合辛普森需要$n=6$,对于高斯求积方法需要$n=13$,将计算出来的理论值代入程序,不同数值积分方法的运行结果,包括$n$的值,对区间的等分数,计算结果,以及误差,均在表\ref{result}中给出。其中step表示区间的等分数。

    \begin{table}[!htbp]
        \centering
        \begin{tabular}{l|llll}
            \hline
            \hline
            方法名称 & n   & val            & err            & step \\
            \hline
            复合梯形 & 476 & 1.718282460433 & 0.000000631974 & 476  \\
            复合辛普森 & 6   & 1.718282288438 & 0.000000459979 & 12   \\
            龙贝格 & 3   & 1.718281828795 & 0.000000000335 & 8    \\
            高斯 & 13  & 3.141592653681 & 0.000000000091 & 13 \\
            \hline
            \hline 
        \end{tabular}
        \caption{运行结果}
        \label{result}
    \end{table}

    原始运行结果在`integral.out'文件中给出。通过误差的计算可以知道,实际中的误差比理论中的误差要小。
    
    \section{体会与展望}
    通过数值积分的实验,知道了一些数值积分的方法,知道了在工程上或者积分解析解足够困难的时候如何去处理积分。做完这个实验中我觉得收获很大。
    
    \end{document}