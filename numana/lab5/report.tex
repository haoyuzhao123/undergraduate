%Homework template created by Haoyu Zhao, Oct 23,2017
\documentclass{article}

\usepackage{ctex}
\usepackage{fontspec}

%the math packages
\usepackage{amsmath,amsfonts,amssymb,amsthm}
\theoremstyle{plain}
\newtheorem{thm}{Theorem}[section]
\newtheorem{lem}[thm]{Lemma}
\newtheorem{prop}[thm]{Proposition}
\newtheorem*{cor}{Corollary}

\theoremstyle{definition}
\newtheorem{defn}{Definition}[section]
\newtheorem{conj}{Conjecture}[section]
\newtheorem{exmp}{Example}[section]

\theoremstyle{remark}
\newtheorem*{rem}{Remark}
\newtheorem*{note}{Note}

\usepackage{mathtools}
\usepackage{optidef}
%add new math packages here

%--------------------------


%the algorithm packages
\usepackage{algorithm}
\usepackage{algorithmicx}
\usepackage{algpseudocode}
\algrenewcommand\algorithmicrequire{\textbf{Input:}}
\algrenewcommand\algorithmicensure{\textbf{Output:}}
%add new algorithm packages here

%-------------------------------

\usepackage{graphicx}
\usepackage{tikz}
\usepackage{subfigure}

\usepackage{listings}

% In case you need to adjust margins:
\topmargin=-0.45in      %
\evensidemargin=0in     %
\oddsidemargin=0in      %
\textwidth=6.5in        %
\textheight=9.0in       %
\headsep=0.25in         %

%newcommand for the format of the homework
\newcommand{\Answer}{\ \\\textbf{Answer:} }
\newcommand{\Acknowledgement}[1]{\ \\{\bf Acknowledgement:} #1}

\newcommand\numberthis{\addtocounter{equation}{1}\tag{\theequation}}
%end the newcommand for this part

%new command for the partial derivatives
\newcommand{\pd}[2]{\frac{\partial #1}{\partial #2}}
\newcommand{\spd}[3]{\frac{\partial^2 #1}{\partial #2 \partial #3}}
\newcommand{\grad}[1]{\nabla #1}
\newcommand{\curl}[1]{\nabla \times #1}
\newcommand{\dive}[1]{\nabla \cdot #1}
%end the new command for this part

\title{\bf\huge 数值分析实验四}
\author{赵浩宇,2016012390,计科60}

\begin{document}

\maketitle

\section{实验要求}
考虑n阶希尔伯特矩阵$H_n$,其元素为$h_{ij}=\frac1{i+j-1}$。

(1)按$\infty-$范数计算$H_3,H_4$的条件数。

(2)令$n=10$,生成Hilbert矩阵,并构造向量$b=H_nx$,其中x是所有分量都是1的列向量,用矩阵三角分解(LU分解)的方法求解以$H_n$作为系数矩阵的线性方程组$H_nx=b$,得到近似解$\hat x$,计算残差$r=b-H_n\hat x$的$\infty-$范数$||r||_{\infty}$,以及误差$\Delta x=\hat x-x$的$\infty-$范数$||\Delta x||_{\infty}$

(3)(选做)由于上述矩阵为对称矩阵,采用平方根法(Cholesky分解)重新求解上述方程,比较其与LU分解方法的运行效率。

(4)让上述线性方程组的右端项b产生$10^{-7}$的扰动,然后重新求解上述方程组,观察得到的解产生的误差的变化情况

(5)减小或增大n的值,观察$||\Delta x||_{\infty}$的变化情况,n取大约多少值时,误差达到$100\%$?

\section{算法描述}
\textbf{1.} 首先求条件数的算法可分为两步,第一步是求出矩阵$A$的逆矩阵$A^{-1}$,第二步则是计算条件数。对于第一步,可以将矩阵写为$[A I]$,之后对这个矩阵进行行变换,对$A$进行高斯消元使其变为$I$,那么复合矩阵变为$[I,B]$,那么$B$就是$A$的逆矩阵。这是因为
\[A^{-1}A = I \Rightarrow A^{-1}I = B \Rightarrow A^{-1} = B.\]
有了矩阵及其逆矩阵,求条件数只需要根据矩阵无穷范数的定义,找出行的绝对值和最大的即可。

\textbf{2.} 用LU分解法解方程的算法需要两步,第一步是对矩阵进行LU分解,第二步则是通过LU分解解出方程。首先LU分解可以使用公式,$U$的第一行和$L$的第一列可以确定。
\[u_{1i} = a_{1i}, l_{i1} = \frac{a_{i1}}{a_11}.\]
之后如果$L,U$的前$r-1$行和$r-1$列已知,那么可以得到
\[u_{ri} = a_{ri} - \sum_{k=1}^{r-1}l_{rk}u_{ki},i = r,\dots,n.\]
同理也可以得到
\[u_{ri} = \frac{a_{ri} - \sum_{k=1}^{r-1}l_{ik}u_{kr}}{u_{rr}},i = r+1,\dots,n.\]
则LU分解可以得到。下面进行第二步,假设我们要求解$Ax = b$而且$A = LU$,那么首先可以求出$y$使得$Ly = b$,这个求解比较容易,之后再求解$x$使得$Ux = y$即可。

\textbf{3.} 与LU分解法差不多,Cholesky可以描述为如下算法
\begin{algorithm}[!htbp]
    \centering
    \begin{algorithmic}[1]
        \Require $A$
        \Ensure $L$,其中$A = LL^T$.
        \Procedure{Cholesky}{$A$}
            \For{$j=1,2,\dots,n$}
                \State $l_{jj} = (a_{jj} - \sum_{k=1}^{j-1}l_{jk}^2)^{\frac{1}{2}}$.
                \State $l_{ij} = (a_{ij} - \sum_{k=1}^{j-1}l_{ik}l_{jk})/l_{jj},i = j+1,\dots,n$.
            \EndFor
        \EndProcedure
    \end{algorithmic}
\end{algorithm}

与LU分解法解方程相同,假设我们要求解$Ax = b$而且$A = LL^T$,那么首先可以求出$y$使得$Ly = b$,这个求解比较容易,之后再求解$x$使得$L^Tx = y$即可。
    
\section{程序清单}
详细程序清单见表\ref{list}。
\begin{table}[!htbp]
    \centering
    \begin{tabular}{c|c}
        \hline
        \hline
        文件名称 & 作用 \\
        \hline
        codes.cc & 实验代码\\
        lab.out & 实验结果输出文件\\
        report.tex & 实验报告源码\\
        report.pdf & 实验报告\\
        \hline
        \hline
    \end{tabular}
    \caption{程序清单}
    \label{list}
\end{table}

\section{运行结果}
\textbf{1.} 计算可以得到,当$n=3$时,无穷范数意义下的条件数为$728$。当$n=4$时,在无穷范数意义下的条件数为$28375$。

\textbf{2.} 下面给出LU分解,Cholesky分解,与加入扰动之后的LU分解得到的$||r||_{\infty},||\Delta x||_{\infty}$,结果在表\ref{p234res}中给出。LU分解与Cholesky分解得到的答案在`lab.out'中给出。同时对比LU分解与Cholesky分解的运行效率,可以知道Cholesky分解效率约为LU分解的一半左右。

\begin{table}[!htbp]
    \centering
    \begin{tabular}{|l|l|l|}
    \hline\hline
    Method     & $||r||_{\infty}$ & $||\Delta x||_{\infty}$ \\ \hline
    LU         & 0.000000000000   & 0.000436911530          \\
    Cholesky   & 0.000000000000   & 0.000444585071          \\
    LU-Perturb & 0.000000000000   & 0.700640814414          \\ \hline
    \end{tabular}
    \caption{不同的计算方法对应的$||r||_{\infty},||\Delta x||_{\infty}$}
    \label{p234res}
\end{table}


\textbf{3.} 当对$n$的值进行调整并且将结果进行输出,结果见表\ref{p5res}。可以看出,当$n=13$时,误差已经超过$100\%$。

\begin{table}[!htbp]
    \centering
    \begin{tabular}{|c|c|c|}
    \hline
    \hline
    $n$ & $||r||_{\infty}$ & $||\Delta x||_{\infty}$ \\
    \hline
        5  & 0.000000000000 & 0.000000000003   \\
        6  & 0.000000000000 & 0.000000000425   \\
        7  & 0.000000000000 & 0.000000015222   \\
        8  & 0.000000000000 & 0.000000487042   \\
        9  & 0.000000000000 & 0.000016869114   \\
        10 & 0.000000000000 & 0.000436911530   \\
        11 & 0.000000000000 & 0.009726486882   \\
        12 & 0.000000000000 & 0.204232243715   \\
        13 & 0.000000000000 & 6.493529661418   \\
        14 & 0.000000000000 & 10.428207690148  \\
        15 & 0.000000000000 & 8.355047567109   \\
        16 & 0.000000000000 & 9.180072412591   \\
        17 & 0.000000000000 & 18.416462283497  \\
        18 & 0.000000000000 & 46.128917552869  \\
        19 & 0.000000000000 & 13.341959391111  \\
        20 & 0.000000000000 & 58.516277170010  \\
        21 & 0.000000000000 & 61.697589597599  \\
        22 & 0.000000000000 & 67.008000514975  \\
        23 & 0.000000000000 & 171.558350683744 \\
        24 & 0.000000000000 & 46.170493102604  \\
        25 & 0.000000000000 & 151.476438012963 \\
        26 & 0.000000000000 & 122.603044468031 \\
        27 & 0.000000000000 & 61.096976107076  \\
        28 & 0.000000000000 & 548.858039252004 \\
        29 & 0.000000000000 & 82.203974368480 \\ \hline\hline
    \end{tabular}
    \caption{不同的$n$对应的$||r||_{\infty},||\Delta x||_{\infty}$}
    \label{p5res}
\end{table}

原始及完整运行结果在`lab.out'文件中给出。
    
\section{体会与展望}
通过进行数值线性代数的实验,更深入的了解了LU分解以及Cholesky分解的算法以及运行效率。并且通过数值稳定性实验,知道了Hilbert矩阵的数值不稳定的性质,将数值分析理论与实践结合起来。

\end{document}